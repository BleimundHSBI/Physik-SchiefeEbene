\section{Messdaten}

Zunächst erfolgt die Darstellung der Messergebnissem um auf deren Basis die Haftreibungskoeffizienten $\mu_{HR}$ und Gleitreibungskoeffizienten $\mu_{GR}$ für die drei Oberflächenmaterialien X, Y, und Z zu bestimmen.

\subsection{Messungen zur Haftreibung}

Im ersten Schritt muss der möglicherweise vorhandene Justierfehler des WInkelmessers bestimmt werden, da dieser, wenn vorhanden, einbezogen werden muss. Dazu kann der Winkelmesser an einen bekannten Winkel angelegt werden und die Differenz berechnet werden. 

Es ergibt sich ein Justierfehler von:
\begin{align*}
    \alpha_{justier} = 0^\circ
\end{align*}

\subsection{Messungen zur Gleitreibung}

\subsubsection{Zurückgelegte Strecke des Körpers}

\subsubsection{Eingestellter Winkel}

\subsubsection{Gemessene Zeit $t_{mess}$}