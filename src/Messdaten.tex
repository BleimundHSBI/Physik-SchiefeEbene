\section{Messdaten}

Zunächst erfolgt die Darstellung der Messergebnissem um auf deren Basis die Haftreibungskoeffizienten $\mu_{HR}$ und Gleitreibungskoeffizienten $\mu_{GR}$ für die drei Oberflächenmaterialien X, Y, und Z zu bestimmen.

\subsection{Messungen zur Haftreibung} \label{chap:MessungHaftreibung}

Im ersten Schritt muss der möglicherweise vorhandene Justierfehler des WInkelmessers bestimmt werden, da dieser, wenn vorhanden, einbezogen werden muss. Dazu kann der Winkelmesser an einen bekannten Winkel angelegt werden und die Differenz berechnet werden. 

Es ergibt sich ein Justierfehler von:
\begin{align*}
    \alpha_{justier} = 0^\circ
\end{align*}

Für den Versuch werden die drei Oberflächenmaterialien X, Z und Z gewählt. Die ... zeigt die Messergebnisse für den maximalen Neigungswinkel $a_{max}$.

\begin{table}[h]
    \center
    \caption[Messung des maximalen Neigungswinkel]{Messung des maximalen Neigungswinkel $a_{max}$ für drei unterschiedliche Oberflächen X, Y und Z}
    \begin{tabular}{c|c|c|c}
    \space & \multicolumn{3}{c}{Winkel $a$ / $^\circ$} \\ \hline
    Messung & Oberfläche X & Oberfläche Y & Oberfläche Z \\
\end{tabular}
    \label{tab:maximalerNeigungswinkel}
\end{table}

\subsection{Messungen zur Gleitreibung}

Nach der Darstellung der Messergebnisse zum Haftreibungskoeffizienten $\mu_{HR}$ erfolgt nun die Darstellung der Messergebnisse zum Gleitreibungskoeffizienten für dieselben drei Oberflächenmaterialien X, Y und Z.

\subsubsection{Zurückgelegte Strecke des Körpers}

Als Vorbereitung muss zunächst die Strecke $l$ zwischen den beiden Lichtschranken bestimmt werden.
Dazu wird mit einem Messstab der Abstand zwischen den Lichtschranken gemessenen.

\begin{align*}
    l = 527 mm = 0,527 m
\end{align*}

Die Messunsicherheit liegt bei dieser Methode bei $\Delta l = \pm X mm$. Für die Strecke $l$ gilt demnach:

\begin{align*}
    l \pm (0,5 + 0,5)mm
\end{align*}

\subsubsection{Eingestellter Winkel}

Um den Versuch durchzuführen, müssen zuvor etwas größere Neigungswinkel $\alpha_i$ eingestellt werden als bei der Bestimmung der jeweiligen Haftreibungskoeffizienten in \autoref{chap:MessungHaftreibung}. Die Neigungswinkel können der \autoref{tab:voreingestellterNeigungswinkel} entnommen werden.

\begin{table}[h]
    \center
    \caption[]{Für den Versuch voreingestellte Neigungswinkel $a_i$}
    \begin{tabular}{c|c|c|c}
    Oberfläche X & Oberfläche Y & Oberfläche Z & Messunsicherheit \\
    $\alpha_1$ / $^\circ$ & $\alpha_2$ / $^\circ$  & $\alpha_3$ / $^\circ$ & $\Delta \alpha$ / $^\circ$ \\ \hline 
    45,8 & 17,3 & 32,0 & 0,9
\end{tabular}
    \label{tab:voreingestellterNeigungswinkel}
\end{table}

\subsubsection{Gemessene Zeit $t_{mess}$}