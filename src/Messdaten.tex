\section{Messdaten}

Zunächst erfolgt die Darstellung der Messergebnissem um auf deren Basis die Haftreibungskoeffizienten $\mu_{HR}$ und Gleitreibungskoeffizienten $\mu_{GR}$ für die drei Oberflächenmaterialien X, Y, und Z zu bestimmen.

\subsection{Messungen zur Haftreibung}

Im ersten Schritt muss der möglicherweise vorhandene Justierfehler des WInkelmessers bestimmt werden, da dieser, wenn vorhanden, einbezogen werden muss. Dazu kann der Winkelmesser an einen bekannten Winkel angelegt werden und die Differenz berechnet werden. 

Es ergibt sich ein Justierfehler von:
\begin{align*}
    \alpha_{justier} = 0^\circ
\end{align*}

Für den Versuch werden die drei Oberflächenmaterialien X, Z und Z gewählt. Die ... zeigt die Messergebnisse für den maximalen Neigungswinkel $a_{max}$.

\begin{table}[h]
    \center
    \caption[Messung des maximalen Neigungswinkel]{Messung des maximalen Neigungswinkel $a_{max}$ für drei unterschiedliche Oberflächen X, Y und Z}
    \begin{tabular}{c|c|c|c}
    \space & \multicolumn{3}{c}{Winkel $a$ / $^\circ$} \\ \hline
    Messung & Oberfläche X & Oberfläche Y & Oberfläche Z \\
\end{tabular}
    \label{tab:maximalerNeigungswinkel}
\end{table}

\subsection{Messungen zur Gleitreibung}

\subsubsection{Zurückgelegte Strecke des Körpers}

\subsubsection{Eingestellter Winkel}

\subsubsection{Gemessene Zeit $t_{mess}$}