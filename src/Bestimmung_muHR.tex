\subsection{Bestimmung des Haftreibungskoeffizienten}

Im Folgenden werden die Berechnungen zur Bestimmung des Haftreibungskoeffizienten $\mu_{HR}$ durchgeführt.

\subsubsection{Statistische Auswertung des Winkels}

Die in \autoref{tab:maximalerNeigungswinkel} gemessenen Werte werden für die weitere Arbeit statistisch Ausgewertet. Dazu wird der Mittelwert $\bar{\alpha}_{max}$, die Standardabweichung der Stichprobe $\sigma_{n-1}$ und die Standardabweichung des Mittelwertes $\sigma_{\alpha}$ berechnet.

In \autoref{tab:statWinkel} sind die Ergebnisse der statischen Auswertung zu sehen.

\begin{table}[h]
    \center 
    \caption[Statistische Auswertung des maximalen Neigungswinkels]{Ergebnisse der statischen Auswertung des maximalen Neigungswinkels}
    \begin{tabular}{c|c|c|c}
    \space & Oberfläche X & Oberfläche Y & Oberfläche Z \\ \hline
    $\bar{\alpha}_{max}$ & 38,94 & 10,10 & 22,86 \\
    $\sigma_{n-1}$ & 1,411737 & 1,009950 & 1,451895 \\
    $\sigma_{\bar{\alpha}}$ & 0,631348 & 0,451664 & 0,649307 \\
\end{tabular}
    \label{tab:statWinkel}
\end{table}

\subsubsection{Berechnung des Haftreibungskoeffizienten und Größtfehlers}

\subsubsection{Angabe des Endergebnisses}