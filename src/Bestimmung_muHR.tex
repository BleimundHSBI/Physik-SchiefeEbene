\subsection{Bestimmung des Haftreibungskoeffizienten}

Im Folgenden werden die Berechnungen zur Bestimmung des Haftreibungskoeffizienten $\mu_{HR}$ durchgeführt.

\subsubsection{Statistische Auswertung des Winkels}

Die in \autoref{tab:maximalerNeigungswinkel} gemessenen Werte werden für die weitere Arbeit statistisch ausgewertet. Dazu wird der Mittelwert $\bar{\alpha}_{max}$, die Standardabweichung der Stichprobe $\sigma_{n-1}$ und die Standardabweichung des Mittelwertes $\sigma_{\alpha}$ berechnet.

In \autoref{tab:statWinkel} sind die Ergebnisse der statischen Auswertung zu sehen.

\begin{table}[h]
    \center 
    \caption[Statistische Auswertung des maximalen Neigungswinkels]{Ergebnisse der statischen Auswertung des maximalen Neigungswinkels}
    \begin{tabular}{c|c|c|c}
    \space & Oberfläche X & Oberfläche Y & Oberfläche Z \\ \hline
    $\bar{\alpha}_{max}$ & 38,94 & 10,10 & 22,86 \\
    $\sigma_{n-1}$ & 1,411737 & 1,009950 & 1,451895 \\
    $\sigma_{\bar{\alpha}}$ & 0,631348 & 0,451664 & 0,649307 \\
\end{tabular}
    \label{tab:statWinkel}
\end{table}

\subsubsection{Berechnung des Haftreibungskoeffizienten und Größtfehlers}

Die Haftreibungskraft $\vec{F}_{HR}$ wirkt dann, wenn sich der Körper nicht bewegt. Da er sich nicht bewegt, kompensiert $\vec{F}_{HR}$ die Hangabtriebskraft $\vec{F}_{HA}$. Die Haftreibungskraft hat einen maximalen Wert von $\vec{F}_{HR}^{max}$, welche Sie nicht überschreiten kann. Daher kann mit \autoref{eq:F_HR}, \autoref{eq:F_HA} und der \autoref{fig:Kräfte}, $\mu_{HR}$ hergeleitet werden.

\begin{align*}
  \abs{\vec{F}_{HR}^{max}} = \abs{\vec{F}_{HA}} &= \mu_{HR} \cdot \abs{\vec{F}_{N}} \quad \text{mit} & \abs{\vec{F}_{HA}} = m \cdot g \cdot \sin(\alpha_{max}) \, , \, \abs{\vec{F}_{N}} &= m \cdot g \cdot cos(\alpha_{max})
\end{align*}
\vspace{-1cm}
\begin{alignat}{3}
  &\Rightarrow &\quad m \cdot g \cdot \sin(\alpha_{max}) &= \mu_{HR} \cdot m \cdot g \cdot cos(\alpha_{max}) \nonumber \\
  &\Leftrightarrow &\quad \frac{m \cdot g \cdot \sin(\alpha_{max})}{m \cdot g \cdot \cos(\alpha_{max})} &= \mu_{HR}\nonumber \\ 
  &\Leftrightarrow &\quad \tan(\alpha_{max}) &= \mu_{HR} \label{eq:muHR}
\end{alignat}
\begin{conditions}
  \alpha_{max} & Der größte Winkel, bei dem der Körper noch ruht.
\end{conditions}

\newpage

Da $\mu_{GR}$ mit gemessenen Werten berechnet wird, muss eine Größtfehlerbetrachtung gemacht werden. Dafür gilt:


\begin{align}
\mu_{\mathrm{HR}} &=\tan \left(\alpha_{\max }\right) \quad \Rightarrow \nonumber \\
\Delta \mu_{\mathrm{HR}}&=\left|\frac{\partial \mu_{\mathrm{HR}}}{\partial \alpha_{\max }} \cdot \Delta \alpha_{\max }\right|=\abs*{\frac{1}{\cos ^{2}\left(\alpha_{\max }\right)} \cdot \Delta \alpha_{\max }} \label{eq:DeltaMuHR}
\end{align}
\begin{conditions}
\alpha_{max} & maximaler Neigungswinkel. Angegeben im Bogenmaß.
\end{conditions}

In \autoref{tab:muHRWerte} wird mit \autoref{eq:DeltaMuHR} und \autoref{eq:muHR}, $\mu_{HR}$ und $\Delta \mu_{HR}$ für alle drei Oberflächen berechnet.

\begin{table}[h]
  \center 
  \caption[Haftreibungskoeffizienten und Größtfehler]{Ergebnisse der Berechnung des Haftreibungskoeffizienten $\mu_{HR}$ und Größtfehlers}
  \begin{tabular}{c|c|c|c}
    \space & Oberfläche X & Oberfläche Y & Oberfläche Z \\ \hline
    $\mu_{GR}$ & 0,808052 & 0,178127 & 0,421594 \\
    $\Delta \mu_{GR}$ & 0,018214 & 0,008133 & 0,013347
\end{tabular}
  \label{tab:muHRWerte}
\end{table}

\subsubsection{Angabe des Endergebnisses}

In \autoref{tab:EndergebnissMuHR} erfolgt die Angabe des Endergebnisses für den Haftreibungskoeffizienten $\mu_{HR}$.

\begin{table}[h]
  \center 
  \caption[Endergebnisse des Haftreibungskoeffizienten]{Angabe des Endergebnisses für den Haftreibungskoeffizienten $\mu_{HR}$}
  \begin{tabular}{c|ccc}
    Oberfläche & $\mu_{HR}$ & $\pm$ & $\Delta \mu_{HR}$ \\ \hline
    X & 0,808 & $\pm$ & 0,019 \\
    Y & 0,178 & $\pm$ & 0,009 \\
    Z & 0,422 & $\pm$ & 0,014
\end{tabular}
  \label{tab:EndergebnissMuHR}
\end{table}