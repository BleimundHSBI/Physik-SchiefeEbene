\subsection{Diskussion}
Einige Variablen haben bei diesem Versuch einen einfluss auf das ergebnis, wobei sich nur wenige davon kontrollieren lassen. Die am einfachsten zu kontrollierende Variable ist die Länge $l$ zwischen den beiden Lichtschranken.
\\
Diese Erhöhung kann zu einem genaueren Ergebnis führen, weil sich somit der Messfehler von $l$ prozentual verringert. Ein weiterer Vorteil von einer höheren Länge ist, dass Fehler, wie eine leichte Startgeschwindigkeit des Körpers, auf die erhöhte Länge betrachtet, einen geringeren Einfluss haben. \\ $l$ kann allerdings auch nicht beliebig angehoben werden, da der Körper nicht immer in einer graden Linie auf der Ebene hinunterrutscht und somit das Ergebnis verfälschen kann. Ebenfalls kann dieses schiefe Rutschen dazu führen, dass der Körper die untere Lichtschranke garnicht trifft und somit garkeine Messung stattfindet und die jeweilige Messung wiederholt werden muss.\bigbreak 
Eine weitere Möglichkeit genauere Ergebnisse zu erhalten ist den Winkel $\alpha$ genauer zu messen und ihn zwischen Messungen genauer zu halten, da es bei dem aktuellen Aufbau sein kann, dass sich $\alpha$ minimal verschiebt. Dies ist ein Problem, da immer mit dem selben Winkel gerechnet wird und eine leichte Verstellung die ohnehin schon hohe Messunsicherheit weiter erhöht. \\ $\alpha$ ist mit einer Messunsicherheit von $\Delta \alpha = 0,9\circ$ hoch und eine Möglichkeit zur genaueren Messung kann ebenfalls einen positiven Einfluss auf die genauigkeit des Ergebnisses haben.