\subsection{Bestimmung des Gleitreibungskoeffizienten}

\subsubsection{Statistische Auswertung der gemessenen Zeit}

Die in \autoref{tab:gemesseneZeit} gemessenen Werte werden für die weitere Arbeit statistisch Ausgewertet. Dazu wird der Mittelwert $\bar{t}_{mess}$, die Standardabweichung der Stichprobe $\sigma_{n-1}$ und die Standardabweichung des Mittelwertes $\sigma_{\bar{t}}$ berechnet.

In \autoref{tab:statZeit} sind die Ergebnisse der statischen Auswertung zu sehen.

\begin{table}[h]
    \center 
    \caption[Statistische Auswertung der gemessenen Zeit]{Ergebnisse der statischen Auswertung der gemessenen Zeit}
    \begin{tabular}{c|c|c|c}
    \space & Oberfläche X & Oberfläche Y & Oberfläche Z \\ \hline
    $\bar{t}_{mess}$ & 0,485730 & 0,813020 & 0,545010 \\
    $\sigma_{n-1}$ & 0,009620 & 0,026040 & 0,019323 \\
    $\sigma_{\bar{t}}$ & 0,003042 & 0,008235 & 0,006111 
\end{tabular}
    \label{tab:statZeit}
\end{table}

\subsubsection{Berechnung des Gleitreibungskoeffizienten und Größtfehlers}

Die Gleitreibungskraft $\vec{F}_{GR}$ wirkt, wenn sich der Körper bewegt. Sie ist kleiner als die Haftreibungskraft $\vec{F}_{HR}$. Aus \autoref{fig:Gesamtkraft} und \autoref{eq:F_N} und \autoref{eq:F_GR} kann somit $\mu_{GR}$ bestimmt werden.

\begin{align}
  \mu_{GR} = \tan{a} - \frac{2\cdot l}{g \cdot t_{mess}^2 \cdot \cos{a}}
  \label{eq:muGR}
\end{align}

Da $l$ und $t_{mess}$ gemessen werden, muss eine Größtfehlerbetrachtung gemacht werden. Dafür gilt:

\begin{align}
  \mu_{GR} = tan(\alpha) - \frac{2 \cdot l}{g \cdot \bar{t}_{mess}^2 \cdot \cos(\alpha)} \quad \Rightarrow
\end{align}
\begin{equation}
  \begin{alignedat}{2}
    \Delta \mu_{GR} &= &\quad \abs*{\frac{\partial \mu_{GR}}{\partial l} \cdot \Delta l}_{\bar{t}_{mess},\alpha = konst.} + \abs*{\frac{\partial \mu_{GR}}{\partial \bar{t}_{mess}} \cdot \Delta \bar{t}_{mess}}_{l,\alpha = konst.} + \abs*{\frac{\partial \mu_{GR}}{\partial \alpha} \cdot \alpha}_{\bar{t}_{mess},l = konst.}\\
    &= &\, \abs*{-\frac{2}{g \cdot \bar{t}_{mess}^2 \cdot \cos(\alpha)}\cdot \Delta l}_{\bar{t}_{mess},\alpha = konst.} + \abs*{\frac{4 \cdot l}{g \cdot \bar{t}_{mess}^2 \cdot \cos(\alpha)} \cdot \Delta \bar{t}_{mess}}_{l,\alpha = konst.} \\ 
    &\quad  &+ \, \abs*{\frac{1}{\cos^2( \alpha )} -\frac{2 \cdot l \cdot \sin(\alpha)}{g \cdot \bar{t}_{mess}^2 \cdot \cos^2(\alpha)} \cdot \Delta \alpha}_{\bar{t}_{mess},l = konst.}
  \end{alignedat}
  \label{eq:DeltaMuGR}
\end{equation}
\begin{conditions}
  \bar{t}_{mess} & Mittelwert der einzelnen Messungen. \autoref{tab:statZeit}\\
  \Delta \bar{t}_{mess} & Standardabweichung des Mittelwertes. \autoref{tab:statZeit}\\
  \alpha & Eingestellter Winkel in Bogenmaß.
\end{conditions}

In \autoref{tab:EndergebnissMuGR} werden die Gleitreibungskoeffizienten $\mu_{GR}$ und Größtfehlerbetrachtung, mit \autoref{eq:muGR} und \autoref{eq:DeltaMuGR}, für alle drei Oberflächen gemacht.

\begin{table}[h]
  \center 
  \caption[Endergebnisse des Gleitreibungskoeffizienten]{Angabe des Endergebnisses für den Gleitreibungskoeffizienten $\mu_{GR}$}
  \begin{tabular}{c|ccc}
    Oberfläche & $\mu_{HR}$ & $\pm$ & $\Delta \mu_{HR}$ \\ \hline
    X & 0,38 & $\pm$ & 0,04 \\
    Y & 0,141 & $\pm$ & 0,021 \\
    Z & 0,198 & $\pm$ & 0,029
\end{tabular}
  \label{tab:EndergebnissMuGR}
\end{table}

\subsubsection{Angabe des Endergebnisses}