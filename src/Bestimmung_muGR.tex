\subsection{Bestimmung des Gleitreibungskoeffizienten}

\subsubsection{Statistische Auswertung der gemessenen Zeit}

Die in \autoref{tab:gemesseneZeit} gemessenen Werte werden für die weitere Arbeit statistisch Ausgewertet. Dazu wird der Mittelwert $\bar{t}_{mess}$, die Standardabweichung der Stichprobe $\sigma_{n-1}$ und die Standardabweichung des Mittelwertes $\sigma_{\bar{t}}$ berechnet.

In \autoref{tab:statZeit} sind die Ergebnisse der statischen Auswertung zu sehen.

\begin{table}[h]
    \center 
    \caption[Statistische Auswertung der gemessenen Zeit]{Ergebnisse der statischen Auswertung der gemessenen Zeit}
    \begin{tabular}{c|c|c|c}
    \space & Oberfläche X & Oberfläche Y & Oberfläche Z \\ \hline
    $\bar{t}_{mess}$ & 0,485730 & 0,813020 & 0,545010 \\
    $\sigma_{n-1}$ & 0,009620 & 0,026040 & 0,019323 \\
    $\sigma_{\bar{t}}$ & 0,003042 & 0,008235 & 0,006111 
\end{tabular}
    \label{tab:statZeit}
\end{table}

\subsubsection{Berechnung des Gleitreibungskoeffizienten und Größtfehlers}

\subsubsection{Angabe des Endergebnisses}