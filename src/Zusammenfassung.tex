\section{Zusammenfassung und Ausblick}

Wie in \autoref{chap:Diskussion} beschrieben wird, gibt es viele Möglichkeiten genauere Ergebnisse für $\mu_{HR}$ und $\mu_{GR}$ zu erhalten. Es ist auch zu erkennen, dass sich die Haftreibungskoeffizienten $\mu_{HR}$ und Gleitreibungskoeffizienten $\mu_{GR}$ mit den unterschiedlichen Oberflächenmaterialien ändern. Dabei kann erkannt werden, dass die Koeffizienten von der Härte des Materials abhängig sind. Gummi hat jeweils die Größten Koeffizienten. Lackiertes Holz, welches härter ist, besitzt die kleinsten Koeffizienten.

Ebenfalls wird erkannt, dass genauere Messergebnisse mit homogenen Oberflächen erzielt werden. Dies lässt sich an dem Besipiel von der mit Klebeband beschichteten Blockseite erkennen. Da das Klebeband uneben abgenutzt ist und somit keine homogene Fläche gegeben ist. Das hat zur Folge, dass der Block nicht gerade an der schiefen Ebene herunterrutscht. Er fängt an schief zu rutschen, da unterschiedliche Reibungen an unterschiedlichen Stellen der Oberfläche wirken. Die Länge ist somit nicht mehr eindeutig zu bestimmen und variiert bei jeder Durchführung.