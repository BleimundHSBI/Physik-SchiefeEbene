\section{Versuchsaufbau}

Für den Versuch wird ein Gerüst benutzt, welches ein glattes Brett hält und ermöglicht den Winkel zwischen der horizontalen Ebene und dem Brett einzustellen. An beiden Enden des Bretts befindet sich jeweils eine Lichtschranke, welche sich verschieben lassen.\\
Ein Holzquader mit vier unterschiedlichen Oberflächen wird auf das glatte Brett gelegt. Eine Seite des Quaders ist unbeschichtet, eine ist mit Gummi beklebt, auf einer ist Klebeband und die letzte Seite ist lackiert. Der Quader rutscht über das Brett, sobald der Winkel hoch genug ist.\\
Ebenfalls wird ein Gliedermaßstab, um die Länge $l$ zwischen den Lichtschranken zu messen, und ein Winkelmesser, mit dem $\alpha$ berechnet wird, gebraucht.
\begin{figure}[ht]
    \centering
    \includegraphics[width=\linewidth/2]{images/Versuch-Aufbau.jpg}
    \caption[Aufbau]{Gerüst mit glattem Brett und Lichtschranken}
    \label{fig:Aufbau}
\end{figure}